% % \part{优化模型}
% % \chapter{线性规划和整数规划}

\chapter{线性规划}

\section{问题的引入与分析}
    \par
    前面,我们讨论的都是非线性规划,无约束非线性规划和约束非线性规划,下面,我们讨论两种特殊情况:1.线性规划2.二次规划。线性规划要求目标函数和约束条件皆为线性函数。二次规划则要求目标函数为二次函数。我们先来讨论线性规划。
    \par
    示例:设某工厂用4种资源生产3种产品,单位$j$产品需要$i$资源的数量为$a_{ij}$,可获利$c_j$。并要求第$i$种资源总消耗不超过$b_i$,第$j$种产品产量不超过$d_j$,问如何安排生产使总利润最大?
    \par
    解:设3种产品的产量分别为$x_1,x_2,x_3$,则
    \begin{align*}
    &{\max}\ \mathop{\sum}\limits_{j=1}^3c_jx_j\\
    &s.t.\left\{
    \begin{aligned}
    &\mathop{\sum}\limits_{j=1}^3a_{ij}x_j \leqslant b_i\quad i=1,2,3,4\\
    &x_j \leqslant d_j\\
    &x_j \geqslant 0\\
    &j=1,2,3
    \end{aligned}
    \right.
    \end{align*}
    \par
    线性规划的目标函数$f$是线性的,约束函数是线性的,约束有等式和不等式两种。Optimization Toolbox采用下列3种方法求解线性规划:
    \par
    \ding{172}单纯形算法,也是最常用的算法;
    \par
    \ding{173}内点算法:基于原始预估校正算法,尤其适用于稀疏结构或其它特殊结构的大规模问题;
    \par
    \ding{174}动态序列算法。
\section{模型规范化及基本理论}
    \par
    线性规划的一般形式为
    \begin{align*}
    & \mathop{\min}\limits_{x\in R^n}\ f(x)=c^\mathrm{T} x\\
    &s.t.\left\{
    \begin{aligned}
    &Ax \leqslant b\\
    &x \geqslant 0
    \end{aligned}
    \right.
    \end{align*}
    其中:$x=(x_1,x_2,\ldots,x_n)^\mathrm{T}\in R^n $,$f=c^\mathrm{T} x:R^n \to R$为线性函数。$A \in R^{m\times n}$,$b \in R^m$,$c \in R^n$。
    \par
    设解的可行域为$D=\{x|Ax \leqslant b\ \& \ x \geqslant 0\}$,最优解记为$x^*$,上面的线性规划问题也可以写成分量形式
    \begin{align*}
    & \min \  \mathop{\sum}\limits_{j=1}^nc_jx_j\\
    &s.t.\left\{
    \begin{aligned}
    &\mathop{\sum}\limits_{j=1}^na_{ij}x_j\leqslant b_i \quad i=1,2,\ldots,m\\
    &x_j \geqslant 0\quad j=1,2,\ldots,n
    \end{aligned}
    \right.
    \end{align*}
    设$E=\{1,2,\ldots,m\}$为指标集,$J=\{1,2,\dots,n\}$。
    \par
    我们可以将上面的线性规划(LP)一般形式化为标准形,标准形的定义如下
    \begin{align*}
    & \mathop{\min}\  c^\mathrm{T} x\\
    &s.t.\left\{
    \begin{aligned}
    &Ax = b\\
    &x \geqslant 0
    \end{aligned}
    \right.
    \end{align*}
    其中:$A \in R^{m\times n},b\in R^m$,$c,x \in R^n$。记可行域为$S$
    \begin{align*}
    S=\{x|x\in R^n|Ax=b,x \geqslant 0\}
    \end{align*}
    \par
    下面,我们将线性规划一般形式转化为标准形式:\\
    1)不等式转化为等式:\par
    对于$\mathop {\sum}\limits_{j=1}^n a_{ij}x_j \leqslant b_i$,增加一个松弛变量
    \begin{align*}
    b_i - \mathop {\sum}\limits_{j=1}^n a_{ij}x_j + r_i\geqslant 0
    \end{align*}
    \par
    对于$\sum\limits_{j=1}^n a_{ij}x_j \geqslant b_i$,增加一个剩余变量
    \begin{align*}
    \sum_{j=1}^n a_{ij}x_j - b_i+s_i \geqslant 0
    \end{align*}
    2)受限与非受限变量转化为非负变量:\par
    对于$x_j \geqslant l_j$,进行平移变换:${\bar{x}}_j = x_j-l_j \geqslant 0$;\par
    对于$x_j \leqslant u_j$,进行反射变换与平移变换:$x_j=u_j-{\bar{x}}_j \geqslant 0$;\par
    对于自变量$x_j \in R$,将它分解成非负变量之差:$x_j={\bar{x}}_j-{\hat{x}}_j$,其中,${\bar{x}}_j \geqslant 0,{\hat{x}}_j \geqslant 0$。\\
    3)极大化转化为极小化目标。\par
    由可行域$S$的定义可知,$S$是一个凸集,事实上,$S$是一个多面体区域。
    % $S$无界的充要条件是它有方向$d \in R^n$是线性规划。可行域$S$的一个方向的充要条件是
    % \begin{align*}
    % d \geqslant 0 \ \& \ Ad=0
    % \end{align*}
    % \par
    设可行域$S$的极点为$x^{i}(i \in E)$,极方向为$d^{j}(j \in J)$,那么对于任意的点$x \in S$,有
    \begin{align*}
    \left\{
    \begin{aligned}
    &x=\mathop{\sum}\limits_{i\in E}{\lambda}_ix^i+\mathop{\sum}\limits_{j\in J}{\mu}_jd^j\\
    &\mathop{\sum}\limits_{i\in E}{\lambda}_i=1\\
    &{\lambda}_i \geqslant 0\\
    &{\mu}_j \geqslant 0
    \end{aligned}
    \right.
    \end{align*}
    把$x$代入原问题,得到以${\lambda}_i,{\mu}_j$为变量的等价的线性规划
    \begin{align*}
    &\min\  \mathop {\sum}\limits_{i\in E}(c^\mathrm{T} x^i){\lambda}_i+\mathop{\sum}\limits_{j\in J}(c^\mathrm{T} d^j){\mu}_j\\
    &s.t.\left\{
    \begin{aligned}
    &\mathop{\sum}\limits_{i\in E}{\lambda}_i=1\\
    &{\lambda}_i \geqslant 0\\
    &{\mu}_j \geqslant 0\\
    &i \in E\\
    &j \in J
    \end{aligned}
    \right.
    \end{align*}
    由于${\mu}_j \geqslant 0$可以任意大。因此,若对于某个$j$有$c^\mathrm{T} d^j<0$,则$(c^\mathrm{T} d^j){\mu}_j$随着${\mu}_j$的增大而无限减小,从而目标函数值趋向$-\infty$,称该问题无界或不存在有限最优值。如果对于所有的$j\in J$,有$c^\mathrm{T}d^j \geqslant 0$,则相对于最小化目标来说,令$\mu_j = 0(j\in J)$,于是,线性规划的标准形式转化为
    \begin{align}
    \label{eq:线性规划的标准形式}
    \min\ \mathop {\sum}\limits_{i\in E}(c^\mathrm{T} x^i){\lambda}_i\\
    s.t.\left\{
    \begin{aligned}
    &\mathop{\sum}\limits_{i\in E}{\lambda}_i=1\\\notag
    &{\lambda}_i \geqslant 0\\\notag
    &i \in E\notag
    \end{aligned}
    \right.
    \end{align}
    \par
    在上述问题中,令
    \begin{align*}
    c^\mathrm{T} x^p=\mathop{\min}\limits_{i}\ c^\mathrm{T} x^i
    \end{align*}
    显然,当${\lambda}_p=1$并且${\lambda}_i=0(i \neq p)$时,目标函数值最小,所以(\ref{eq:线性规划的标准形式})式必然有最优解。
    \paragraph{线性规划的基本定理}
    假设线性规划标准形式的可行域$S\neq \phi$,则有\par
    (1)标准形存在有限最优解,当且仅当,对于$S$的任意极方向$d^j(j \in J)$,有
    \begin{align*}
    e^\mathrm{T} d^j \geqslant 0
    \end{align*}
    \par
    (2)若标准形存在有限最优解,则其最优值可以在$S$的某个极点上取到。
\section{线性规划的最优化条件}
    \par
    最优化条件 - KKT条件。对于一般形式的线性规划而言,$x^* \in R^n$是其最优解,当且仅当存在向量$w \in R^m,r \in R^n$使得
    \begin{align}
    \label{eq:线性规划的最优化条件1}
    &Ax^* \geqslant b,x^* \geqslant 0 \notag \\
    &c-A^\mathrm{T} w-r = 0,w \geqslant 0,r \geqslant 0
    \end{align}
    和
    \begin{align}
    \label{eq:线性规划的最优化条件2}
    w^\mathrm{T} (Ax^*-b) = 0,r^\mathrm{T} x^*=0
    \end{align}
    \par
    对于标准形式的线性规划而言,$x^* \in R^n$是其最优解,当且仅当存在向量$w \in R^m,r \in R^n$,使得
    \begin{align*}
    &Ax^* = b \quad x^* \geqslant 0 \\
    &A^\mathrm{T} w+r =c \quad r \geqslant 0 \\
    &r^\mathrm{T} x^*=0
    \end{align*}
    最优性条件将求解线性规划的问题转化为求解代数方程组(不等式组)的问题,后者有$n+m+1$个变量和$n+m+1$个方程。
\section{对偶理论}
    \par
    在线性规划的KKT条件中,条件方程(\ref{eq:线性规划的最优化条件1})(\ref{eq:线性规划的最优化条件2})分别等价于下面的不等式组和方程组
    \begin{align*}
    &A^\mathrm{T} w \leqslant c \quad w \geqslant 0 \\
    &b^\mathrm{T} w - (w^\mathrm{T} A)x^*=0 \quad c^\mathrm{T} x^*- w^\mathrm{T} (Ax^*)=0
    \end{align*}
    于是,我们可以写出如下形式的对偶规划
    \begin{align*}
    & \mathop {\max}\  b^\mathrm{T} w\\
    & s.t.\left\{
    \begin{aligned}
    & A^\mathrm{T} w \leqslant c\\
    & w \geqslant 0
    \end{aligned}
    \right.
    \end{align*}
    我们称上述规划为原线性规划的对偶形式(DLP)。
    \paragraph{弱对偶定理}
    (1)设原线性规划的可行域为$S$,对偶规划的可行域为$T$,若$S \neq \phi,T \neq \phi$,则$\forall  x \in S,w \in T$,有$c^\mathrm{T} x \geqslant b^\mathrm{T} w$。
    (2)若$\exists x^* \in S,w^* \in T$,使得$c^\mathrm{T} x^*=b^\mathrm{T} w^*$,则$x^*,w^*$分别为PLP和DLP的最优解。
    (3)若PLP无下界,则其对偶DLP是不相容的(即$T=\phi$),反之,若DLP无上界,则PLP是不相容的(即$S=\phi$)。
    \paragraph{强对偶定理}
    (1)若PLP和DLP中任何一个问题存在有限的最优解,则另一个问题也存在有限的最优解,并且它们的目标函数最优值相等。
    (2)对于PLP或DLP问题,若目标函数值无界,则另一个问题是不相容的(无可行解)。
\section{最优化算法}
    \subsection{内点算法}
        \par
        考虑标准形式的线性规划问题
        \begin{align}
        \label{eq:标准形式的线性规划问题}
        & \mathop {\min}\  b^\mathrm{T} y\\
        & s.t.\left\{
        \begin{aligned}
        & A^\mathrm{T} y = c\\\notag
        & y \geqslant 0\notag
        \end{aligned}
        \right.
        \end{align}
        其中:$A \in R^{m\times n},b,y \in R^m,c \in R^n$。
        上述问题的对偶问题为
        \begin{align}
        \label{eq:对偶问题的线性规划问题}
        & \mathop {\max}\  c^\mathrm{T} x\\
        & s.t.\quad Ax \leqslant b\notag
        \end{align}
        其中:$c,x \in R^n , A \in R^{m\times n},m \geqslant n$。
        \par
        \underline{先假设存在内点$x_0$},并假设问题是有界的。内点法的基本思想是:从内点$x_0$出发,沿可行方向求出使目标函数值上升的后继点,再从得到的内点出发,沿另一个可行方向求使目标函数值上升的内点,重复以上步骤,产生一个内点组成的序列$\{x_k\}$,使得
        \begin{align*}
        &c^\mathrm{T} x_{k+1} > c^\mathrm{T} x_k=0
        \end{align*}
        当满足终止准则时,则停止迭代。这种方法的关键是选择使得目标函数值上升的可行方向。
        \par
        首先,引进松弛变量$v$,将模型(\ref{eq:对偶问题的线性规划问题})写为标准型
        \begin{align*}
        &\mathop {\max}\  c^\mathrm{T} x\\
        &s.t.\left\{
        \begin{aligned}
        & Ax + v = b \\
        & v \geqslant 0
        \end{aligned}
        \right.
        \end{align*}
        在第$k$次迭代,定义$v_k$为非负松弛变量构成的$m$维向量,使得
        \begin{align*}
        v_k = b - Ax_k
        \end{align*}
        再定义对角矩阵
        \begin{align*}
         D_k = diag \left( \frac{1}{v_1^k},\cdots,\frac{1}{v_m^k} \right)
        \end{align*}
        作仿射变换,令
        \begin{align*}
         w = D_kv
        \end{align*}
        把线性规划(\ref{eq:对偶问题的线性规划问题})改写为
        \begin{align*}
        &\mathop {\max}\  c^\mathrm{T} x\\
        &s.t.\left\{
        \begin{aligned}
        & Ax+D_k^{-1}w = b\\
        & w \geqslant 0
        \end{aligned}
        \right.
        \end{align*}
        在变换空间中,选择搜索方向
        \begin{align*}
         d = \begin{bmatrix}d_x\\d_w\end{bmatrix}
        \end{align*}
        显然,$d$作为可行方向,它必是下列齐次方程的一个解
        \begin{align}
        \label{eq:其次方程的一个解}
         D_kAd_x+d_w=0
        \end{align}
        对于上式的任一解,有
        \begin{align*}
         A^\mathrm{T} D_k(D_kAd_x+d_w)=0
        \end{align*}
        由此得到
        \begin{align*}
         d_x=-(A^\mathrm{T} D_k^2A)^{-1}A^{T}D_kd_w
        \end{align*}
        每次迭代中,目标函数在$d_x$方向的方向导数是
        \begin{align*}
         c^\mathrm{T} d_x
        \end{align*}
        将$d_x$代入上式,则有
        \begin{align*}
         c^\mathrm{T} d_x=c^\mathrm{T} [-(A^\mathrm{T} D_k^2A)^{-1}A^{T}D_kd_w]=-[D_kA(A^\mathrm{T} D_k^2A)^{-1}c]^\mathrm{T} d_w
        \end{align*}
        选择$d_w$,使$c^\mathrm{T} d_x$最大,则
        \begin{align*}
         d_w=-D_kA(A^\mathrm{T} D_k^2A)^{-1}c
        \end{align*}
        由上式确定$d_w$后,可以得到式(\ref{eq:其次方程的一个解})中的一个解,其中
        \begin{align*}
         d_x=(A^\mathrm{T} D_k^2A)^{-1}c
        \end{align*}
        同时,对$d_w$作逆仿射变换,可得到
        \begin{align*}
         d_v=D_k^{-1}d_w=-A(A^\mathrm{T} D_k^2A)^{-1}c=-Ad_x
        \end{align*}
        搜索方向确定后,还需确定沿此方向移动的步长。设后继点
        \begin{align*}
         x_{k+1}=x_k+\alpha d_x
        \end{align*}
        步长$\alpha$在保证$x_{k+1}$为可行域的内点情况下取值,即满足
        \begin{align*}
        &A(x_k+\alpha d_x) < b\\
        &{\alpha}Ad_x < b-Ax_k\\
        &-\alpha d_v < v_k
        \end{align*}
        \par
        令
        \begin{align*}
        \beta = {\min}\bigg \{\frac{v_i^{(k)}}{-(d_v)_i}\bigg |(d_v)_i < 0,i \in\{1,2,\ldots,m\}\bigg\}
        \end{align*}
        取$\alpha=\gamma\beta$,其中$\gamma \in (0,1)$,这样即可得到$x_{k+1}$。下面给出内点法的计算步骤:\\
        \textbf{step1.}初始化。
        $x_0 \in R^n$,$\gamma \in (0,1)$,容许误差$\varepsilon > 0$,置$k:=0$\\
        \textbf{step2.}计算$v_k$
        \begin{align*}
        v_k=b-Ax_k
        \end{align*}
        \textbf{step3.}置对角矩阵\\
        \begin{align*}
        D_k = diag \left( \frac{1}{v_1^k},\cdots,\frac{1}{v_m^k} \right)
        \end{align*}
        \textbf{step4.}计算$d_x=(A^\mathrm{T} D_k^2A)^{-1}c$\\
        \textbf{step5.}令$d_v=-Ad_x$\\
        \textbf{step6.}令
        \begin{align*}
        \alpha = \gamma\ {\min}\bigg \{\frac{v_i^{(k)}}{-(d_v)_i}\bigg |(d_v)_i < 0,i \in\{1,2,\ldots,m\}\bigg\}
        \end{align*}
        \textbf{step7.}置$x_{k+1}=x_k+\alpha d_x$\\
        \textbf{step8.}若$\frac{|c^\mathrm{T} x_{k+1}-c^\mathrm{T} x_k|}{c^\mathrm{T} x_k} < \varepsilon$则停止,输出$x_{k+1}$;否则,置$k:=k+1$,返回step2。
        \par
        前面,我们假设存在内点$x_0$。这里,我们可以如此求初始内点:
        首先,从原点出发,沿目标函数的梯度方向$c$取一点,令
        \begin{align*}
        x_0 = \left( \frac{\|b\|}{\|A^c\|} \right) _c
        \end{align*}
        如果$v_0=b-Ax_0>0$,则$x_0$为初始内点,否则,解下列一阶线性规划问题
        \begin{align*}
        & \mathop{\max}\  c^\mathrm{T} x-Mx_a\\
        &s.t.\quad Ax-x_ae \leqslant b
        \end{align*}
        其中:$M$是大的正数,$e$为$1\times m$的单位列向量,$x_a$为人工变量,根据$v$的定义,如果令
        \begin{align*}
        x_a^{(0)} > \left| {\min} \left\{{v_i^{(0)}}\big|i = 1,2,\ldots,m\right\}\right|
        \end{align*}
        则有
        \begin{align*}
        Ax_0- x_a^{(0)}e < b
        \end{align*}
        因此,$(x_0,x_a^{0})$必为内点。
    \subsection{路径跟踪法}
        \par
        考虑线性规划原问题(PLP)
        \begin{align*}
        & \mathop{\min}\ c^\mathrm{T} x\\
        &s.t.\left\{
        \begin{aligned}
        &Ax = b\\
        &x \geqslant 0
        \end{aligned}
        \right.
        \end{align*}
        其对偶形式(DLP)为
        \begin{align*}
        &\mathop {\max}\  b^\mathrm{T} y\\
        &s.t.\left\{
        \begin{aligned}
        & A^\mathrm{T} y+w = c\\
        & w \geqslant 0
        \end{aligned}
        \right.
        \end{align*}
        其中:$c,x \in R^n$,$b,y \in R^m$,$A \in R^{m \times n}$,$Rank(A)= m$。记可行域分别为$S,T$
        \begin{align*}
        &S=\{x|Ax=b,x \geqslant 0\}\\
        &T=\left\{\begin{pmatrix}
        y\\w
        \end{pmatrix}\bigg|A^\mathrm{T} y+w=c,w \geqslant 0\right\}
        \end{align*}
        可行域内部记为$S^\mathrm{T} ,T^\mathrm{T} $
        \begin{align*}
        &S^\mathrm{T} =\{x|Ax=b,x>0\}\\
        &T^\mathrm{T} =\left\{\begin{pmatrix}
        y\\w
        \end{pmatrix}\bigg|A^\mathrm{T} y+w=c,w > 0\right\}
        \end{align*}
        $x,y,w$为最优解的充分必要条件是
        \begin{align*}
        \left\{
        \begin{aligned}
        & Ax = b \quad x \geqslant 0\\
        & A^\mathrm{T} y+w=c \quad w \geqslant 0\\
        & XWe = 0
        \end{aligned}
        \right.
        \end{align*}
        其中:$X=diag(x_1,x_2,\ldots,x_n)$,$W=diag(w_1,w_2,\ldots,w_n)$,上述条件为KKT条件。
        \par
        现在,将$XWe=0$换作$XWe=\mu e$,$e$为$n \times 1$的全1列向量,实参数$\mu > 0$,得到松弛KKT条件
        \begin{align*}
        &Ax=b \quad x\geqslant 0\\
        &A^\mathrm{T} y+w=c \quad w \geqslant 0\\
        &XWe = \mu e
        \end{align*}
        \par
        如果$S$有界且$S^\mathrm{T}  \neq \phi$,则对每一个$\mu $,松弛KKT条件\underline{存在唯一内点解。}
        \begin{definition}[原始 - 对偶中心路径]
        原始 - 对偶可行解$D=\{(x,y,w)|Ax=b,A^\mathrm{T} y+w=c,(w,x) \geqslant 0\}$和可行集内部$D^+$,若$D^+ \neq \phi$,则对每一个$\mu > 0$,上述系统存在唯一解$\left(x(\mu),y(\mu),w(\mu)\right)$,把$\{x(\mu),y(\mu),w(\mu)|\mu > 0\}$称为原始 - 对偶中心路径,记为$C_{\mu}$。
        \end{definition}
        \par
        在中心路径$C_{\mu}$上,当$\mu$很小时,原问题的目标值单调减小且趋于最优值。对偶问题目标值单调增加且趋于最优值。对于每一个中心路径参数$\mu$,对偶间隙$c^\mathrm{T} x(\mu)-b^\mathrm{T} y(\mu)=n\mu$。
        \par
        关于参数$\mu$的确定:如果点$(x,y,w)$在中心路径$C_{\mu}$上,显然有
        \begin{align*}
        \mu = \frac{x^\mathrm{T} w}{n}
        \end{align*}
        如果点$(x,y,w)$不在中心路径$C_{\mu}$上,我们仍用上述方法确定。
        下面介绍如何确定转移方向$d_k$。
        \par
        当$\mu \to 0$时,原始问题和对偶问题均趋于最优值。我们通过迭代,大致沿着$C_{\mu}$去逼近最优解。任取一点$(x,y,w)$,其中,$x>0,w>0$。此时,目标是求一个方向$(\nabla x,\nabla y,\nabla w)$使迭代点$(x+\nabla x,y+\nabla y,w+\nabla w)$位于$C_{\mu}$上,即
        \begin{align*}
        &A(x+\Delta x)=b \\
        &A^\mathrm{T} (y+\Delta y)+(w+\Delta w)=c \\
        &(X+\Delta x)(W+\Delta w)e = \mu e
        \end{align*}
        整理后,有
        \begin{align*}
        &A\Delta x=b-Ax \\
        &A^\mathrm{T} \Delta y+\Delta w=c-A^\mathrm{T} y- A^\mathrm{T} w\\
        &W\Delta x+X\Delta w+\Delta x \Delta w e = \mu e-XWe
        \end{align*}
        记作$b-Ax=\rho$,$c-A^\mathrm{T} y-w=\sigma$,忽略二次项$\Delta x \Delta w$,用矩阵形式表示,则有
        \begin{align*}
        \begin{bmatrix} A & 0 & 0\\
        0 & A^{\mathrm{T}} & I\\ W & 0 & X \end{bmatrix}\begin{bmatrix}\Delta x\\\Delta y\\\Delta w \end{bmatrix}=\begin{bmatrix}\rho \\ \sigma \\\mu e-XWe \end{bmatrix}
        \end{align*}
        \par
        解上述方程,可求出移动方向$[\Delta x,\Delta y,\Delta w]^\mathrm{T} $。在求出转移方向之后,需要确定此方向移动的步长$\alpha$,$\alpha$取值应满足
        \begin{align}
        \label{eq:移动步长的取值要求}
        &x+\alpha \Delta x > 0 \\
        &w+\alpha \Delta w > 0 \notag
        \end{align}
        由于$x_j>0,w_j>0,\alpha >0$,因此
        \begin{align*}
        \frac{1}{\alpha}=\mathop {\max}\limits_{i,j}\left\{ -\frac{\Delta x_j}{x_j},-\frac{\Delta w_i}{w_i}\right\}
        \end{align*}
        为保证(\ref{eq:移动步长的取值要求})式为严格不等式,引进小于且接近$T$的正数$\rho$,令
        \begin{align*}
        {\alpha}=\mathop {\min}\left\{\rho\left[ \mathop {\max}\limits_{i,j} \left( -\frac{\Delta x_j}{x_j},-\frac{\Delta w_i}{w_i} \right)\right]^{-1},1\right\}
        \end{align*}
        \par
        算法流程:\\
        \textbf{step1.}初始化。
        初始点$(x,y,w)$,$x_1>0,w_1>0$,$\rho \to 1$,容许误差$\varepsilon > 0$,正数$M < \infty$,置$k:=1$,$\delta = \frac {1} {10}$\\
        \textbf{step2.}计算$\rho=b-Ax_k,\sigma=c-A^\mathrm{T} y_k-w_k,r=x_k^\mathrm{T} w_k,\mu=\delta \frac \gamma n$。\\
        \textbf{step3.}若$\|\rho\|_1<{\varepsilon \ \& \  \|\sigma}\|_1<\varepsilon\ \&\  r<\varepsilon$则停止迭代,输出解;若$\|x_k\|>M$或者$\|y_k\|>M$则停止迭代,原问题或对偶问题无界,否则,转到step4。\\
        \textbf{step4.}解方程
        \begin{align*}
        \begin{bmatrix} A & 0 & 0\\
        0 & A^\mathrm{T} & I\\ W & 0 & X \end{bmatrix}\begin{bmatrix}\Delta x\\\Delta y\\\Delta w \end{bmatrix}=\begin{bmatrix}\rho \\ \sigma \\\mu e-XWe\end{bmatrix}
        \end{align*}
        得到$(\Delta x_k,\Delta y_k,\Delta w_k)$,置$\alpha$。\\
        \textbf{step5.}计算$x_{k+1}$
        \begin{align*}
        &x_{k+1}=x_k+{\alpha}\Delta x_k \\
        &y_{k+1}=y_k+{\alpha}\Delta y_k\\
        &w_{k+1} =w_k+{\alpha}\Delta w_k
        \end{align*}
        置$k:=k+1$,转到step2。
\section{MATLAB应用实例}
    \par
    MATLAB中用linprog求解线性规划,其调用格式为
    \par
    [x,fval,exitflag,output]=linprog(fun,A,b,Aeq,beq,lb,ub,x0,options)\\
    其中:fun为目标函数;x0为初始点;A,b为$Ax \leqslant b$;Aeq,beq为线性等式约束$Aeq{}x < beq$;lb,ub为$lb \leqslant x \leqslant ub$;nonlcon为非线性约束条件;options为结构体参数。
    \par
    用linprog求解如下线性规划问题
    \begin{align*}
    &\mathop {\min} f=-4x_1-x_2\\
    &s.t.\left\{
    \begin{aligned}
    &-x_1+2x_2 \leqslant 4\\
    &2x_1+3x_2 \leqslant 12\\
    &x_1-x_2 \leqslant 3\\
    &x_1x_2 \geqslant 0
    \end{aligned}
    \right.
    \end{align*}
    \begin{lstlisting}[language=Matlab]
    f=[-4;-1];
    x0=[0,0];
    A=[-1,2;2,3;1,-1];
    b=[4;12;3];
    Aeq=[];
    beq=[];
    lb=[];
    ub=[];
    [x,fval,exitflag,output,lambda]=linprog(f,A,b,Aeq,beq,lb,ub,x0)
    \end{lstlisting}
