% % \part{优化模型}
% % \chapter{多目标规划}

\chapter{多目标规划}
\section{问题的引入与分析}
    考虑在二次规划中引入的组合投资问题
    \begin{align*}
    &\mathop {\min}\  x^\mathrm{T} Qx\\
    &\mathop {\max}\  r^\mathrm{T} x\\
    &s.t.\left\{
    \begin{aligned}
    &\mathop {\sum} x=1\\
    &0 \leqslant x \leqslant 1
    \end{aligned}
    \right.
    \end{align*}
    可以发现,上面的引例中有两个优化目标,这种多个目标的规划问题即为多目标规划(MOP)。
\section{模型规范化及其理论化}
    \par
    多目标规划(multi-objective programming, MOP)是在变量满足给定约束的条件下,研究多个目标函数同时极小化的问题,其一般形式为
    \begin{align*}
    &\mathop {\min}\  f(x)=(f_1(x),f_2(x),\ldots,f_p(x))^\mathrm{T} \\
    &s.t.\left\{
    \begin{aligned}
    &h_i(x) = 0 \quad i \in E\\
    & g_j(x) \geqslant 0 \quad j \in I
    \end{aligned}
    \right.
    \end{align*}
    其中:$x \in R^n$为决策变量(或称优化变量或自变量)。$g_j(x)$和$h_i(x)$均为约束函数,$I,E$为指标集。$I=\{1,2,\ldots,m\},E=\{1,2,\ldots,n\}$,$f_i(x):R^n \to R$,$f$为向量值目标函数。记MOP的可行域为
    \begin{align*}
    S=\{x|x \in R^n,g_j(x)\geqslant 0,h_i(x)=0,i \in E,j \in I\}
    \end{align*}
    \par
    我们把$S$的像集$z=f(S)$称为目标可行域,即因变量域。$z$中元素$z=f(x)$称为目标向量,其中分量$z_i=f_i(x)$是第$i$个目标值。
    若每个$f_i(i=1,2,\ldots,p)$都是凸函数,并且可行域$S$是凸集,则MOP称为多目标凸规划问题。
    \par
    T.C.Koopmans(1951)在其关于数量经济学的工作中,从生产与分配的活动分析的角度提出了多目标规划问题,并且第一次提出了Pareto解的概念。H.W.Kuhn和A.W.Tucker(1951)从数学规划的角度给出了Pareto解的概念,并研究了这种解的充分必要条件,奠定了多目标规划的理论基础。后来,L.Hurwicz(1958)将多目标规划的研究推广到一般的拓扑向量空间中。L.A.Zadeh(1963)从控制论的角度提出了多目标控制问题。
    \par
    多目标规划的3个基本研究方向:(1)解的概念和性质,从最早研究的有效集、弱有效集以及1951年Kuhn-Tucker提出的真有效集开始,迄今,有一定影响的最优解的概念不少于20种。对于每一种解的概念,可以考虑解的存在性、最优性条件、解的连通性与稳定性、以及解与解之间的关系等;(2)多目标规划的对偶问题,已存在多种多样的对偶形式。比如Lagrange对偶,共轭对偶等;(3)不可微多目标规划。已经有多种关于导数的推广概念可以应用于这方向的研究,比如广义梯度、Dini次微分等。
    \par
    在讨论MOP最优解的概念之前,先引入下面的记号:$\forall x,y \in R^n,xy \in z$为多目标函数值
    \begin{align*}
      & x=y \Leftrightarrow x_i=y_i \quad i \in \{1,2,\ldots,n\}\\
      & x \prec y \Leftrightarrow x_i \leqslant y_i \quad i \in \{1,2,\ldots,n\}
    \end{align*}
    但是,存在某个$j$,使得$x_j<y_j$
    \begin{align*}
      & x\succ y \Leftrightarrow y-x \prec 0\\
      & x \leqslant y \Leftrightarrow x_i \leqslant y_i \quad i \in \{1,2,\ldots,n\}\\
      & x\prec y \Leftrightarrow x_i < y_i \quad i \in \{1,2,\ldots,n\}
    \end{align*}
    \subsection{Pareto最优解}
        \par
        下面,给出Pareto最优解的概念。
        给定一个可行点$x^* \in S$,若$\forall x \in S$,有$f(x^*)\leqslant f(x)$,则称$x^*$为MOP的绝对最优解;若不存在$x \in S$,使得$f(x)\prec f(x^*)$,则称$x^*$为MOP的有效解;若不存在$x \in s$,使得$f(x)< f(x^*)$,则称$x^*$为MOP的弱有效解。
        \par
        MOP的有效解通常也称为Pareto最优解。绝对最优解、有效解、弱有效解的集合分别记为$S_a,S_p,S_{wp}$。
        \begin{theorem}
        对于MOP问题,令$z=f(S)$,$z$的有效点集和弱有效点集分别记为$Z_p$和$Z_{wp}$,则(MOP)的有效解集$S_p$和弱有效解集$S_{wp}$由下面式子给出:
        \par
        (1)\ $S_p=\mathop{\bigcup}\limits_{f^*\in Z_p}\{x\in S|f(x)=f^*\}$
        \par
        (2)\ $S_{wp}=\mathop{\bigcup}\limits_{f^*\in Z_{wp}}\{x\in S|f(x)=f^*\}$
        \end{theorem}
        \par
        至于解集$S_a,S_p,S_{wp}$之间的关系,我们有:
        \par
        (1)\ $S_a \subseteq S_p\subseteq S_{wp} \subseteq S$
        \par
        (2)当$S_{a} \neq \phi$时,$S_a = S_p$
        \par
        (3)若$S$为凸集,$f$是$S$上严格凸的向量值函数,则$S_p=S_{wp}$

    \subsection{KT-有效集与G-有效集}
        \par
        MOP的绝对最优解、有效解和弱有效解等概念分别是通过关系$\le,\prec,<$来描述的。如果将序关系的应用范围加以限定,就可以得到其他形式的最优解概念。比如,$\exists x^* \in S,\delta > 0$,使$x^*$是$f(x)$在变量可行域的子集$S\cap N(x^*,\delta)$上的一个有效解,则称$x^*$是一个局部有效解,相应地,$z^*=f(x^*)$称为目标可行域$z=f(S)$的局部有效点。
        \begin{definition}
        对MOP问题,对于$x^* \in S$,令积极不等式约束指标集$I(x^*)=\{i|g_i(x^*)=0\}$,若$x^*$是MOP的有效解,并且不等式组
        \begin{align*}
        \left\{
        \begin{aligned}
        &\nabla f^\mathrm{T}(x^*)x & < 0\\
        &\nabla g_{I(x^*)}^\mathrm{T}(x^*)x & \geqslant 0\\
        &\nabla h^\mathrm{T}(x^*)x& =0
        \end{aligned}
        \right.
        \end{align*}
        在$R^n$中无解,其中:$\nabla f^\mathrm{T} ,\nabla g_{I(x^*)}^\mathrm{T} $和$\nabla h^\mathrm{T} $分别是向量值函数$f(x),g_{I(x^*)}(x)$和$h(x)$的Jacobi矩阵,则称$x^*$为MOP的Kuhn-Tucker真有效解。
        \end{definition}
        \par
        类似地,若$x^*$是MOP的弱有效解,并且不等式组
        \begin{align*}
        \left\{
        \begin{aligned}
        &\nabla f^\mathrm{T}(x^*)x & < 0\\
        &\nabla g_{I(x^*)}^\mathrm{T}(x^*)x & \geqslant 0\\
        &\nabla h^\mathrm{T}(x^*)x& =0
        \end{aligned}
        \right.
        \end{align*}
        在$R^n$中无解,则$x^*$称为MOP的Kuhn-Tucker真弱有效解。
        \begin{definition}
        假设$x^* \in S$是MOP的有效集,若存在$M>0$,使得对于任意的下标$i(1\leqslant i \leqslant p),\{f_i(x)<f_i(x^*)\}$和$x\in S$,必存在下标$j$,使得$f_j(x)>f_j(x^*)$,并且
                \begin{align*}
                f_i(x^*)-f_i(x)\leqslant M(f_j(x)-f_j(x^*))
                \end{align*}
        则称$x^*$为MOP的Geoffrion-真有效集,记其集合为$S_p^G$。并且若$S_a\neq \phi$,则$S_a=S_p^G=S_p$。
        \end{definition}
\section{最优性条件}
    \par
    首先,定义MOP的Lagrange函数如下:
            \begin{align*}
            L(x,\beta,\lambda,\mu)={\beta}^\mathrm{T} f(x)-{\lambda}^\mathrm{T} g(x)-{\mu}^\mathrm{T} h(x)
            \end{align*}
            其中:$\beta \in R^p,{\lambda} \in R^{|I|},{\mu}\in R^{|E|}$。
    \begin{theorem}[Fritz John必要条件]
    假设向量值函数$f,g,h$在$x^*$处可微,若$x^*$是MOP的有效解或弱有效解,则存在向量$\beta \in R_{+}^P,\lambda \in R_{+}^{|I|},\mu \in R_{+}^{|E|}$,使得$(\beta,\lambda,\mu)\neq 0$,并且
    \begin{align}
      & {\nabla}_xL(x^*,\beta,\lambda,\mu)={\nabla}f(x^*){\beta}-{\nabla}g(x^*){\lambda}-\nabla h(x^*)\mu=0    \label{Fritz John必要条件1}\\
      & {\lambda}^\mathrm{T} g(x^*) = 0 \label{Fritz John必要条件2}
    \end{align}
    其中:$\nabla f,\nabla g,{\nabla} h$分别表示向量值函数在相应点的梯度矩阵(即Jacobi矩阵的转置)。为了保证目标向量函数的梯度矩阵$\nabla f$对应的参数$\beta \neq 0$,我们需要对约束函数增加一些限制。
    \par
    \ding{172}线性约束规格(LCQ):$g_j,h_i$为线性函数;
    \par
    \ding{173}Mangasarian-Fromowitz约束规格(MFCQ):矩阵$\nabla h(x^*)$列满秩。
    \par
    \ding{174}线性独立约束规格(LICQ):向量组$\{{\nabla}_{g_i}(x^*),{\nabla}_{h_j}(x^*), i\in I(x^*),j\in E\}$是线性无关的。
    \end{theorem}
    对于$x\in S$以及$d \in R^n$,若向量$d$满足
    \begin{align*}
    & d^\mathrm{T} {\nabla}{g_j}(x) \geqslant 0\quad j \in I(x)\\
    & d^\mathrm{T} {\nabla}{h_i}(x) = 0\quad i \in E
    \end{align*}
    其中:$I(x)$是积极不等式约束的指标集,则称向量$d$为$g,h$在$x$处的一个线性化可行方向。
    \par
    约束$g,h$在$x$点的所有线性化可行方向的集合,称为线性化可行方向锥,记为$LFD(x,s)$。对于$x\in S,d\in R^n$,若存在一个向量序列$\{d^k\}$和正数序列$\{{\delta}_k\}$,使得对$\forall k=1,2,\cdots$,有$x+{\delta}_kd^k \in S$并且$d^k \to d,{\delta}_k \to 0$,则称$d$为$S$在$x$处的一个序列化可行方向。集合$S$在$x$点的所有序列化可行方向的集合,称为序列化可行方向锥,记为$SFD(x,s)$。
    \begin{theorem}[KKT必要条件]
    假设$f,g,h$在$x^*\in S$处可微,若$x^*$是MOP的有效解或弱有效解,并且在$x^*$点\underline{Kuhn-Tucker}约束规格$LFD(x^*,s)=SFD(x^*,s)$成立,则存在非零向量$\beta \in R_{+}^p$以及${\lambda} \in R^{|I|},\mu \in R^{|E|}$并且满足式(\ref{Fritz John必要条件1})(\ref{Fritz John必要条件2})。
    \end{theorem}
    \begin{theorem}[KKT充分条件]
    假设$f,-g$是凸的,在$x^*\in S$处可微,并且$h$是线性函数,若存在非零向量$\beta \in R_{+}^p$和${\lambda} \in R^{|I|},\mu \in R^{|E|}$满足式(\ref{Fritz John必要条件1})(\ref{Fritz John必要条件2}),则$x^*$是MOP的弱有效解。特别地,当$\beta > 0$时,$x^*$是MOP的有效解。
    \end{theorem}

\section{最优化算法}
    对MOP而言,计算所有的最优解是困难的,因为确定整个有效解集的问题是NP难的。下面,介绍一些处理MOP问题的思想。
    \subsection{线性加权和法}
        \par
        根据$p$个目标函数$f_j$的重要程度,赋予各自一定的权重${\lambda}_j$,然后将所有目标函数$f_j$乘上权重${\lambda}_j$求和,变为单目标函数
            \begin{align*}
              & \mathop{\min}\limits_{x\in R^n} \ {\lambda}^\mathrm{T} f(x)=\mathop{\sum}\limits_{j} {\lambda}_jf_j(x)\\
              & s.t.\left\{
                \begin{aligned}
              & g(x)\geqslant 0\\
              & h(x)=0
                \end{aligned}
                 \right.
            \end{align*}
        其中:$f,g,h$为向量值函数。\\
        注:$\forall \lambda >0,\mathop{\sum}\limits_{j} {\lambda}_j=1$,单目标问题的最优解是MOP的弱有效解。
    \subsection{主要目标法}
        \par
        根据实际情况,首先确定一个目标函数为主要目标(例$f_1(x)$),而把其余$p-1$个目标函数$f_j(x)$作为次要目标,然后,再对次要目标选取一定的界限值${\mu}_j(j=2,\cdots,p)$,将其转化为约束条件,例
        \begin{align*}
          & \mathop{\min}\limits_{x\in R^n}\  f_1(x)\\
          & s.t.\left\{
            \begin{aligned}
          & g(x)\geqslant 0\\
          & h(x)=0\\
          & f_j(x)\leqslant {\mu}_j\\
          & j=2,3,\cdots,p
            \end{aligned}
             \right.
        \end{align*}
        注:单目标优化问题的最优解都是MOP的弱有效解。
    \subsection{极小化极大法}
        \par
        极小化极大法的基本思想是,在$f(x)$的$p$个分量中,极小化$f(x)$的最大分量,即
        \begin{align*}
           \mathop{\min}\limits_{x\in S} \mathop{\max}\limits_{1\leqslant j \leqslant p} f_j(x)
        \end{align*}
        该问题的最优解作为MOP的弱有效解。一般地,可以引入目标函数权向量$\lambda :\lambda \geqslant 0,\mathop{\sum}_{j}{\lambda}_j=1$
        \begin{align*}
           \mathop{\min}\limits_{x\in S} \mathop{\max}\limits_{1\leqslant j \leqslant p} {\lambda}_jf_j(x)
        \end{align*}
        注:$\forall \lambda \geqslant 0,\mathop{\sum}\limits_{j} {\lambda}_j=1$单目标问题的最优解都是MOP的弱有效解。
    \subsection{理想点法}
        \par
        对每个目标函数$f_j(x)$,事先确定一个目标值$f_j^0$,其中
        \begin{align*}
           f_j^0=\mathop{\min}\limits_{x\in S} f_j(x)
        \end{align*}
        记理想点为$f^0=(f_1^0,\cdots,f_p^0)^\mathrm{T} $。然后,求解单目标优化问题
        \begin{align*}
           \mathop{\min}\limits_{x\in S} \|f(x)-f^0\|_{\alpha}
        \end{align*}
        注:对于$\alpha \geqslant 1$单目标问题的最优解是MOP的有效解。
    \subsection{分层排序法}
        \par
        根据目标的重要程度将它们一一排序,然后,分别在前一个目标的最优解集中,寻找后一个目标的最优解集,并把最后一个目标的最优解集作为MOP的最优解。
        例如:首先,通过求解单目标优化问题
        \begin{align*}
           \mathop{\min}\limits_{x\in S} f_1(x)
        \end{align*}
        得到最优解集$S^1$,然后,对于$j=2,3,\cdots,p$,依次求解单目标优化问题
        \begin{align*}
           \mathop{\min}\limits_{x\in S^{j-1}} f_j(x)
        \end{align*}
        得到最优解集$S^j$,然后,将$S^j(j=P)$中的点作为MOP的最优解。
\section{MATLAB应用实例}
    \par
    MATLAB可以使用gamultiobj命令求解多目标规划问题,并且gamultiobj使用多目标遗传算法IENSGA(\rom{2})求解多目标问题。gamultiobj的命令格式如下:
    \par
    [x,fval]=gamultiobj(fitnessfcn,nvars,A,b,Aeq,beq,lb,ub,options)\\
    其中:fitnessfun为目标函数;nvars为变量个数;A,b为$Ax\leqslant b$;Aeq,beq为$Aeq {}x= beq$;lb,ub为$lb \leqslant x \leqslant ub$。
    下面,我们介绍函数gamultiobj中的一些基本概念:\\
    \ding{172}个体、种群、代、选择、交叉、变异和交叉后代比例等放在GA章节中介绍;\\
    \ding{173}支配(dominate)与非劣(non-inferior):
    在多目标优化问题中,如果解$x_1 \in R^n$至少有一个目标比解解$x_2 \in R^n$好,而且个体$x_1$的所有目标都不比解$x_2$差,那么称解$x_1$支配解$x_2$,或者$x_1$非劣于$x_2$。\\
    \ding{174}拥挤距离(crowding distance):拥挤距离是用来计算某前端中的某个解$x$与该前端中其它解之间的距离,用以表征解之间的拥挤程度,且只有处于同一前端的解之间才需要计算拥挤程度。\\
    \ding{175}最优前端个体系数(Pareto Fraction):最优前端个体系数定义为最优前端中的解在种群中所占的比例。
    \par
    作为MATLAB的应用实例,多目标遗传算法如下多目标规划问题
    \begin{align*}
       & {\min}\quad f_1=x_1^4-10x_1^2+x_1x_2+x_2^4-x_1^2x_2^2\\
       & {\min}\quad f_2=x_2^4-x_1^2x_2^2+x_1^4+x_1x_2\\
       & s.t.\quad \left\{\begin{aligned}
       & 5 \leqslant x_1 \leqslant 5\\
       & -5 \leqslant x_2 \leqslant 5
       \end{aligned}
       \right.
    \end{align*}
    求解程序如下
    \begin{lstlisting}[language=Matlab]
    function f = multiobj(x)
      f(1) = x(1)^4-10*x(1)^2+x(1)*x(2)+x(2)^4-(x(1)^2)*(x(2)^2);
      f(2) = x(2)^4-(x(1)^2)*(x(2)^2)+x(1)^4+x(1)*x(2);
    end
    fitnessfcn = @multiobj
    nvars = 2;
    lb = [-5,-5];
    ub = [5,5];
    A=[]; b=[];
    Aeq=[]; beq=[];
    options = gaoptimset('ParetoFraction',0.3,'PopulationSiza',100,'Generations',200,'StallGenLimit',200,'TolFun',le-100,'PlotFcns',@gaplotpareto);
    [x, fval] = gamultiobj(fitnessfcn,nvars,A,b,Aeq,beq,lb,ub,options);
    \end{lstlisting}
    \par
    在上面的程序中,我们将IENSGA(\rom{2})算法设置为:最优前端个体系数ParetoFraction为0.3,种群大小PopulationSiza为100,进化代数Generations为200,停止代数StallGenLimit为200,适应度函数值偏差TolFun为le-100,绘制Pareto前端。

