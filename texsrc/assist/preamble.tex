
\ctexset{
    part/number = \chinese{part}
}
%% 符号及公式设置
\usepackage{multirow}
\usepackage{amsmath} % ams 数学公式
\usepackage{amsfonts} % ams 数学字体
\usepackage{bbm} % 重影字体
\usepackage{amssymb,latexsym} % ams 数学符号与LaTeX数学符号
\usepackage{mathrsfs} % 花式符号

%% 定理引理证明
\usepackage{ntheorem} % 定理、定义、证明
    \theoremstyle{nonumberplain}
    \theoremheaderfont{\bfseries}
    \theorembodyfont{\normalfont}
    \theoremsymbol{$\square$}
    \newtheorem{Proof}{\hskip 2em 证明}
    \newtheorem{theorem}{\hspace{2em}定理}[chapter]
    \newtheorem{definition}{\hspace{2em}定义}[chapter] % 如果没有章, 只有节, 把上面的[chapter]改成[section]
    \newtheorem{axiom}[definition]{\hspace{2em}公理} % 类似地定义其他“题头”. 这里“注”的编号与定义、定理等是分开的
    \newtheorem{lemma}[definition]{\hspace{2em}引理}
    \newtheorem{proposition}[definition]{\hspace{2em}命题}
    \newtheorem{corollary}[definition]{\hspace{2em}推论}
    \newtheorem{remark}{\hspace{2em}注}
    \newtheorem{Assumption}{\hspace{2em}假设}[chapter]
    \newtheorem{example}{\hspace{2em}示例}[chapter]

%% 算法伪代码
%http://blog.csdn.net/lwb102063/article/details/53046265
\usepackage{algorithm}
\usepackage{algorithmicx}
\usepackage{algpseudocode}
    \floatname{algorithm}{算法}
    \renewcommand{\algorithmicrequire}{\textbf{输入:}}
    \renewcommand{\algorithmicensure}{\textbf{输出:}}

%%  罗马数字:示例:\rom{2}
\makeatletter
\newcommand*{\rom}[1]{\expandafter\@slowromancap\romannumeral #1@}
\makeatother

%% 文本下划线
\usepackage{ulem}

%% 列表环境
\usepackage{enumerate} % itemiz环境。\begin{enumerate}[step 1][a)]可以使用 A,a,I,i,1 作为可选项产生 \Alph,\alph,\Roman,\roman,\arabic 的效果
\usepackage{enumitem} % enumerate宏包的升级

%% 参考文献(上标)
\usepackage{cite} % 参考文献
    \bibliographystyle{plain}
    \makeatletter
    \def\@cite#1#2{\textsuperscript{[{#1\if@tempswa , #2\fi}]}}
    \makeatother

%% 圆圈①②,并设置圆圈脚注
\usepackage{pifont} % 然后在正文输入\ding{172}~\ding{211}得到相应数字,要是要①就输入:\ding{172}②就输:\ding{173}
\renewcommand\thefootnote{\ding{\numexpr171+\value{footnote}}}

%% 超链接
%\usepackage[CJKbookmarks, colorlinks, bookmarksnumbered=true,pdfstartview=FitH,linkcolor=black,citecolor=black]{hyperref}%超链接的格式设置
\usepackage{hyperref} % 超链接
\hypersetup{
    colorlinks=false, % 去掉超链接颜色
    pdfborder=0 0 0 % 取消超链接的边框
}

%% 图片
\usepackage{graphicx} % 图片管理
\usepackage{caption}
\usepackage{subcaption} %并排的图各有标题
\graphicspath{{images/}} % 设置图片搜索路径
\usepackage{float,varwidth} % 浮动体
\usepackage{caption} % 对标题进行控制,如让\caption标题的字体缩小一号,同时数字标签使用粗体可以用:\usepackage[font=small,labelfont=bf]{caption}

%% 表格
\usepackage{booktabs} % 三线表
\usepackage{tabularx} % 提供自动延伸的表列,(X列格式说明符),文字过长时可以自动转行
\usepackage{longtable} % 长表格
\setlength{\abovecaptionskip}{4pt} % 图片表格的前后距离设置
\setlength{\belowcaptionskip}{-8pt}

\usepackage{extarrows} % 带参数的箭头
\usepackage{xfrac,upgreek} % 分别是行间公式如a/b的形式(将原来的命令\frac改成\sfrac)和希腊字体的宏包的
\usepackage{mathtools} % lgathered和rgathered环境把公式向左向右对齐
\usepackage{harpoon} % 数学公式的矢量
\usepackage{bookmark} % 目录的书签


%% 代码高亮
\usepackage{listings}
\usepackage{xcolor} % 颜色宏包
\usepackage{colortbl} % 彩色表格
\definecolor{codegreen}{rgb}{0,0.6,0}
\definecolor{codegray}{rgb}{0.5,0.5,0.5}
\definecolor{codepurple}{rgb}{0.58,0,0.82}
\definecolor{backcolour}{rgb}{0.95,0.95,0.92}
\lstset{
    commentstyle=\color{codegreen},
    keywordstyle=\color{magenta},
    numberstyle=\tiny\color{codegray},
    stringstyle=\color{codepurple},
    basicstyle=\footnotesize,
    breakatwhitespace=false, % 断行只在空格处
    breaklines=true, % 自动断行
    captionpos=b, % 标题位置
    keepspaces=true,
    numbers=left,
    numbersep=5pt,
    showspaces=false,
    showstringspaces=false,
    showtabs=false, % 显示
    tabsize=2 % TAB 被当作两个空格
}
%% 调用PDF文件
\usepackage{pdfpages}
%% 页面设置
\usepackage{geometry}
\geometry{paperwidth=185mm,paperheight=260mm,left=15mm,right=15mm, bottom=10mm}
% \topmargin=0pt\oddsidemargin=0pt\evensidemargin=0pt
\raggedbottom % 让某些部分更紧凑

%% 页眉页脚设置
\usepackage{fancyhdr} % 页眉设置
%%\pagestyle{headings}
\pagestyle{fancy}
\fancyhead[LE,RO]{\slshape \rightmark}
\fancyhead[LO,RE]{\slshape \leftmark}
\fancyfoot[C]{\thepage}                                          
\fancyfoot[L]{\url{http://www.ma-xy.com}}
\fancyfoot[R]{\url{http://www.ma-xy.com}}
\renewcommand{\footrulewidth}{0pt}     

%% 章节格式设置
\CTEXsetup[format={\zihao{-3}\raggedright\bfseries}]{section} % 设置节的格式
\renewcommand{\headwidth}{\textwidth} % 图片并排,这个要列在所有宏包的后面

%% 水印
\usepackage{draftwatermark} 	%水印draft
\SetWatermarkText{http://www.ma-xy.com}
\SetWatermarkScale{0.5}
%\SetWatermarkAngle{45}
%\SetWatermarkColor{black!60!cyan!20}
%\SetWatermarkFontSize{4cm}


