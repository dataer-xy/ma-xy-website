
{\kaishu
\begin{center}
\LARGE 寒鸦赋其二(渔父答)
\end{center}
\par
{\centering 志士沉吟汨罗间,归棹舟急似水滥,索将身投千尺浪,何事渔父唱空船。

}
\par
晚岁暮薄,乘风临江渚,兴扁舟之一叶,风拂襟而徜徉。至时,暮霭沉沉兮郁结忧思,黯云凝结兮壅塞我心,髣髴兮彤云拂月而蔽日,泬廖兮秋风飘絮而回薄。层波鳞鳞,水波不兴,击水长空,拍石响岸。寂寞兮回首,伤亡兮断肠。不禁郁悒侘傺,哀而叹曰:闻太白之大道,青天兮我独不得出,路难行兮声悲切。屈原既放,披发行吟,贤才暮薄无投处,他生空落白云间。
\par
秋气阴兮烟水沉,碧云落兮远岫岚。江水回波静无声,孤雁回芦怨早寒。三千白发缘愁长,五千立仞行路难。行路难,行路难,九曲三折碎平川,风削雨打如刀环,飞沙激石驰百草,怒卷西风拍云还。
\par
时闻江舟歌曰:“世既浊兮逐波逝,众既醉兮餔糟醨,天道兮已没,人道兮何察,溷溷兮恶恶,沧浪清浊任漂泊。”
\par
余乃问曰:“若随流扬波而成恶,贤者怀才而莫之见用,志未行而心死体伤,不如蹈屈子之迹,行悲吟之调,抱石投河,葬身鱼腹。”
\par
答曰:“道之道也,非常之道。夫天道有常,万物既运,守四时之法,循世逝之理,岂不知道有常,不可为而莫之强求。 ”
\par
回曰:“屈心而抑志,使我兮不如无生。怀情兮不发,余焉能忍而与此终古。”
\par
答曰:“凤兮,凤兮,何德之衰。往者不可谏,来者犹可追,且随天命而安乐,又何忧患之有哉!”
\par
噫,壮志凌云无投地,而今重操梁甫吟。声声悲切无休期,张公龙剑合有时。冀北匹马群逐空,往来寒鸦茕孑飞。天下澄清情为志,欲坠凌云揭重霄。长风破浪济沧海,中流击砫誓不还。蛟龙元非池中物,永嘉枯木今复燃。位卑忧国宵不寐,天下忧乐今始悲。定远笔着无投处,李广射虎响南山。祖刘闻鸡夜起舞,犹有壮心流百川。山河百二关中好,浊漻妙理恐难尽。正江涵秋影雁初飞,醉酒白头今始归。自荐歃血惟一人,肉鄙列曹鼓三关。青天明月往来送,明日乘风破浪还。老矣兮堪惜,回首兮云中。吐尽胸中豪气,荡涤洗尘寰。匣剑穷吟十年,今朝试磨练。剑气清风两刃,冷霜并刀寒。锥骨悬梁志,束修过庭言。
\par
知己书此儿何怠,汉书下酒道称快。悲歌击柱响千载,凭高酹酒云水天。投鞭断流堑南北,横槊赋诗月羞敛。万里乘得鸿鹄去,黄金金甲月光残。剪烛吴钩腰龙泉,刀戈剑铤鸣清泉。长空则虚弦落雁,风飖则一贯三雕。田横烟洲五百士,冒顿鸣镐血沾污。臣死且不避,卮酒安足辞。千里狼烟魄兮归,大刀弯弓向明月。情直上,三万里九天摩千仞,志犹在,下北海蛟龙腾风雨。千秋万载志如此,声鸣于皋鹤九天。

}

